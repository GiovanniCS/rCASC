\title{Stability Score Algorithm}
\author{
      Luca Alessandr\i'
}
\date{\today}

\documentclass[12pt]{article}
\usepackage{amsmath}

\begin{document}
\setcounter{page}{35}



\paragraph{Stability score Algorithm}
Let be $C$ the count matrix with $N \times M $ dimension where N is the gene number and M is cell number.
\[
C = \begin{bmatrix} 
    c_{11} & c_{12} & \dots \\
    \vdots & \ddots & \\
    c_{N1} &        & c_{NM} 
    \end{bmatrix}
\]

Then we define $C^p_q$ as a matrix derived by matrix $C$ in which $q$ columns (randomly selected in permutation $p$) are removed.
The function $RemoveCell(p)$ takes as input a specific permutation $p$ and returns a list $L^p$ containing all the removed cells in $p$(observe that the size of $L^p$ is q).
We denote  $\textbf{cl}^p$ the vector with length $M-q$ storing the output of the clustering algorithm in the permutation $p$. Hence, $\textbf{cl}^p[i]$ identified the cluster in which the i$^{th}$ cell is inserted in permutation p.
Moreover we use the notation  $\textbf{cl}$ for indicating the  the output of the clustering algorithm obtained by all cells.


The relation symmetric matrix $R^p$  with dimension $M \times M$ is defined as follows: 

\[
R^p = \begin{bmatrix} 
    r^p_{1,1} & r^p_{1,2} & \dots \\
    \vdots & \ddots & \\
    r^p_{M,1} &        & r^p_{M,M} 
    \end{bmatrix}
\]

where $r^p_{i,j}$ is:

 \[
   r^p_{i,j}=\left\{
                \begin{array}{ll}
                 1 \ if \ \textbf{cl}^p[i]=\textbf{cl}^p[j]\\
                 0 \ otherwise 
                \end{array}
              \right.
  \]
  
Similarly we defined  $R$   relation symmetric matrix obtained considering  all  cells.
Then we introduce a function $RMV()$ that takes as input $R$ and the list of removed cells in   a permutation (i.e. $l^p$) and returns an new matrix $R'$ in which the columns and the rows associated with cells in  $l^p$ are removed.

Function \emph{length}($j,p,k$) counts  the occurrences of $k$ in the row $j$ of the matrix  
 $R + R^p$. It is formally defined as follows:
  \[
   \it{length}(j,p,k)=\sum_i=1^{M-|L^p|} 1_{R[i,j]} + R^p[i,j]=k
   \]
where $1_A$ is an indicator function returning 1 when condition $A$ is satisfied.

Finally we define a permutation score \emph{pscore}$_{j,p}$ as:
  $$pscore_{j,p}= \frac{ length(j,p,2)}{length(j,p,2)+length(j,p,1)}$$ 
where $p$ is a permutation and j a cell.

We define \emph{tscore}$_{m,s}$ as follow  
$$
\it{tscore}_{m,s}= \frac{1}{P} \sum_{p\in P} 1_{\emph{score}_{m,p}>=s}
$$
where $P$ is the permutation number and s the threshold score.

\section{Example}

Be $\textbf{cl}= \begin{Bmatrix}
 1 & 2 & 2 & 1 & 2 & 1
\end{Bmatrix}  $

Be $\textbf{L}= \begin{Bmatrix}
 6 & 2 & 2 & 4  
\end{Bmatrix}  $

Be $\textbf{cl}^1= \begin{Bmatrix}
 1 & 2 & 1 & 1 & 2 
\end{Bmatrix}  $

Be $\textbf{cl}^2= \begin{Bmatrix}
 1 & 2 & 1 & 2 & 2 
\end{Bmatrix}  $

Be $\textbf{cl}^3= \begin{Bmatrix}
 1 & 2 & 1 & 1 & 2 
\end{Bmatrix}  $

Be $\textbf{cl}^4= \begin{Bmatrix}
 1 & 2 & 2 & 1 & 2 
\end{Bmatrix}  $

$\forall \ p \in \{1,2,3,4\}, \ R_p $ is calculated \\
for istance hereafter I reported R and $R^1$
 

$ R=\begin{bmatrix}
1 & 0 & 0 & 1 & 0 & 1 \\
0 & 1 & 1 & 0 & 1 & 0\\
0 & 1 & 1 & 0 & 1 & 0\\
1 & 0 & 0 & 1 & 0 & 1\\
0 & 1 & 1 & 0 & 1 & 0\\
1 & 0 & 0 & 1 & 0 & 1
\end{bmatrix}  $

$ R^1=\begin{bmatrix}
1 & 0 & 1 & 1 & 0 \\
0 & 1 & 0 & 0 & 1 \\
1 & 0 & 1 & 1 & 0 \\
1 & 0 & 1 & 1 & 0 \\
0 & 1 & 0 & 0 & 1 
\end{bmatrix}  $ \\ \\


$ \forall \ p \ R'+R^p \ is \ calculated$ \\ 
for istance hereafter I reported R' + $R^1$ \\ \\


$ R'+R^1=\begin{bmatrix}
2 & 0 & 1 & 2 & 0 \\
0 & 2 & 1 & 0 & 2 \\
1 & 1 & 2 & 1 & 1 \\
2 & 0 & 1 & 2 & 0 \\
0 & 2 & 1 & 0 & 2 
\end{bmatrix}  $ 


$ \forall \ p ,\ pscore \ is \ evaluated \ with \ S=0.6$
for istance hereafter I reported $pscore_1$ \\ \\


$ pscore_1=\begin{bmatrix}
1 \\
1 \\
0 \\
1 \\
1 
\end{bmatrix}  $
\\ This means that in permutation 1 the third cell is unstable "jumping" from cluster number 1 to cluster number 2 in $P$.
tscore$_{m,s}$ is evaluated then for each cell
$ tscore_s=\begin{bmatrix}
0 \\
0 \\
0 \\
0.25 \\
0.25 \\
0 
\end{bmatrix}  $

In this Example number of permutation is 4, for statistical relevance, an higher number of permutation is required. 
\bibliographystyle{abbrv}
\bibliography{main}

\end{document}
This is never printed